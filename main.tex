\documentclass[12pt]{article}

-\usepackage{sbc-template}

\usepackage{graphicx,url}

\usepackage[brazil]{babel}
\usepackage[utf8]{inputenc}


\sloppy

\title{ Revolução Verde, Biotecnologia e Teconolgias Alternativas. }

\author{ Leonardo Tironi Fassini }


\address{Instituto de Informática -- Universidade Federal da Fronteira Sul (UFFS)\\
  Chapecó -- SC -- Brasil
  \email{leehtironi@gmail.com }
}

\begin{document}

\maketitle

\begin{abstract}
  This article is intended to sintetize the article ``\emph{Revolução Verde, Biotecnologia e Teconolgias Alternativas}''\cite{livro}, relating it whith the videos seen in class, besides criticizing the subject matter.

\begin{resumo}
  Este artigo tem como o intuito sintetizar o artigo ``\emph{Revolução Verde, Biotecnologia e Teconolgias Alternativas}''\cite{livro}, relacionando-o com vídeos vistos em aula, além de críticas ao assunto abordado.
\end{resumo}


\section{ Introdução }

A Revolução Verde foi uma promessa para aumentar a oferta de alimentos, proporcionando a erradicação da fome\cite{albergoni} e tem como intuito aumentar a produtividade e a qualidade nutricional do cultivo, através de pesquisas em agrotóxicos, em modificação genética das sementes, dos sistemas de produção e outras tecnologias\cite{livro}.

Grande quantidade de cientistas se uniram para pesquisar e desenvolver sementes (de trigo, milho e feijão, primariamente) que pudessem ser cultivadas em qualquer época do ano, contanto que houvesse condições favoráveis ao seu cultivo, e.g. temperatura e água acima de uma quantidade mínima.

\section{ Críticas ao Sistema de Produção }

Tendo em vista o alto uso de agrotóxicos e a destruição do solo, oriundos desse novo sistema de produção, nascem e crescem críticas a respeito do quão benéfico para os próprios humanos e para o ambiente o uso destes produtos, uma vez que seus impactos podem afetar gerações.

As críticas giram em torno de três pontos principais. A primeiro, já citado anteriormente, é a destruição do ambiente. A segunda tem como base o fato das políticas publicas favorecerem as elites. Por fim, o terceiro ponto é no sentido econômico, devido ao aumento do custo do pacote tecnológico da Revolução Verde.

\section{ Biotecnologia }

A biotecnologia é um papel importante na sociedade atual, pois é usada para aumento de produção, de valor nutricional, além de diminuir os custos de produção. Ela é utilizada principalmente para o melhoramento genético, através de métodos especializados, tais como técnicas de engenharia genética e Biologia Molecular.

No quesito negativo, a biotecnologia vegetal é ainda um método relativamente obscuro, pois ainda não existe uma definição certa do que a modificação genética dessas plantas causa ao corpo humano a longo prazo, seja no quesito de toxinas ou alergias.

Ademais, existe também uma incerteza no que pode ser criado além dessas sementes modificadas, tais como criação de novas plantas e ervas daninhas e como evitar a poluição tanto de plantações vizinhas, como a biodiversidade daquela região.

Por fim, a criação de monopólios e oligarquias de produtores de sementes genéticamente modificadas também é uma preocupação, uma vez que a propriedade intelectual pode acabar se concentrando em pequenos grupos de empresas.

\section{ Tecnologias Alternativas }

Existem tecnologias alternativas a esse modo de produção, uma vez que ele é totalmente focado na maximização do lucro e minimização das perdas e custos. Esse método alternativo, chamado de Agricultura Alternativa tem como objetivo realizar a produção de um modo menos tóxico ao meio ambiente, através de métodos de produção que não sejam tão agressivos ao solo, focado na cultura de mais de um tipo de produto e que esse produto tenha uma boa qualidade.

Dentre as tecnologias que tem propostas divergentes ao atual modelo de produção agrícola, a Agricultura Biodinâmica é uma das principais. Ela tem como objetivo transformar a empresa agrícola em um organismo agrícola, sendo que todos os recursos necessários para a sua produção já estão dentro dela própria como uma autorregulagem, além de um foco na manutenção do solo.

No mesmo sentido da Agricultura Biodinâmica, surge a Agricultura Orgânica. Seu ponto principal está em manter um equilíbrio ecológico a fim de ciclar ao máximo os nutrientes já existentes na proriedade. É inadmissível a utilização de adubos minerais e outros métodos químicos para adubagem da terra.

Outro método de produção é a Agricultura natural, que tem como objetivos não cultivar o solo, nem utilizar fertilizantes químicos ou orgânicos, nem capinar o solo ou utilizar agrotóxicos. Eles tem como padrões utilizarem-se de rotações de plantio, adubações verde, cobertura morta melhoria das condições do solo através de implantações de inimigos naturais.

\bibliographystyle{sbc}
\bibliography{sbc-template}

\end{document}
