\documentclass[12pt]{article}

-\usepackage{sbc-template}

\usepackage{graphicx,url}

\usepackage[brazil]{babel}
\usepackage[utf8]{inputenc}


\sloppy

\title{ Revolução Verde, Biotecnologia e Teconolgias Alternativas. }

\author{ Leonardo Tironi Fassini }


\address{Instituto de Informática -- Universidade Federal da Fronteira Sul (UFFS)\\
  Chapecó -- SC -- Brasil
  \email{leehtironi@gmail.com }
}

\begin{document}

\maketitle

\begin{abstract}
  This article is intended to sintetize the article ``\emph{Revolução Verde, Biotecnologia e Teconolgias Alternativas}''\cite{livro}, relating it whith the videos seen in class, besides criticizing the subject matter.

\begin{resumo}
  Este artigo tem como o intuito sintetizar o artigo ``\emph{Revolução Verde, Biotecnologia e Teconolgias Alternativas}''\cite{livro}, relacionando-o com vídeos vistos em aula, além de críticas ao assunto abordado.
\end{resumo}


\section{ Introdução }

A Revolução Verde foi uma promessa para aumentar a oferta de alimentos, proporcionando a erradicação da fome\cite{albergoni} e tem como intuito aumentar a produtividade e a qualidade nutricional do cultivo, através de pesquisas em agrotóxicos, em modificação genética das sementes, dos sistemas de produção e outras tecnologias\cite{livro}.

Grande quantidade de cientistas se uniram para pesquisar e desenvolver sementes (de trigo, milho e feijão, primariamente) que pudessem ser cultivadas em qualquer época do ano, contanto que houvesse condições favoráveis ao seu cultivo, e.g. temperatura e água acima de uma quantidade mínima.

\subsection{ Críticas ao Sistema de Produção }

Tendo em vista o alto uso de agrotóxicos e a destruição do solo, oriundos desse novo sistema de produção, nascem e crescem críticas a respeito do quão benéfico para os próprios humanos e para o ambiente o uso destes produtos, uma vez que seus impactos podem afetar gerações.

\subsection{ Visão Holística do Risco }

Esta seção trata de como cada empresa é diferente da outra, cada qual com suas peculiaridades, como se fosse um corpo humano, onde a informação é o sangue, ou seja o que mantém o corpo vivo.

Essa analogia é usada para refletir sobre como nos anos recentes as informações estão cada vez menos centralizadas e mais dispersas, com várias formas de acessá-las. Contudo, aumentando a descentralização, também aumenta a chance de acontecer alguma coisa com os dados, o que é um risco a ser tomado.

\subsection{ Receita Explosiva }

Neste capítulo o autor expõe como os problemas atuais influenciam negativamente no negócio, de modo que dificulta a gestão dele.

\subsection{ Ciclo de Vida da Informação }

A presente seção trata de como a informação deve ser tratada, desde o recebimento da informação até o seu descarte, com um enfoque maior no descarte e em como essa informação deve ser tratada corretamente para evitar riscos.

\section{ Desafios }

\subsection{ Anatomia do Problema }

Este capítulo trata de como o problema é muito maior do que as pessoas geralmente imaginam, pois pensam que apenas ataques de fora ou problemas nos cabos são os perigos, quando existem muitos outros.

O autor também afirma que deve-se saber de quais fatores possuem algum risco para o negócio, listando-os e em caso de problemas, verificar qual a causa mais provável e tratá-la de maneira correta, não extrapolando mas também não fraca o suficiente para que não haja risco de não consertar o problema.

Também é tratado que cada empresa, devido as suas peculiaridades devem ter tratamentos de segurança diferenciados e personalizados para cada tipo de negócio, entretanto nunca haverá uma segurança total.

\subsection{ Visão Corporativa }

Esta seção trata de como podemos ver as portas que é deixada pelo usuário, sendo suas seguranças mal distribuídas, pois o fato de superproteger uma, pode deixar a outra a mercê de algum invasor. Assim, é necessário que existam planos de segurança para evitar que o trabalho esteja sendo feito com alto risco ou com grandes vulnerabilidades. O capítulo após este (\emph{pecados praticados}) trata dos problemas da falta de visão no quesito de manobras incorretas feitas, acreditando que não afetará a segurança.

\subsection{ Conscientização do Corpo Executivo }

A presente seção explica que é necessário que haja uma conversa, não só com o baixo escalão mas também com as pessoas que estão no topo da empresa, com o intuito de viabilizar projetos de segurança, para que tenha um apoio não só a percepção do problema, mas a distribuição de recursos compatíveis com a gravidade dele.

\section{ Retorno Sobre o Investimento}

É uma ferramenta que serve para ajudar a decidir o rumo do que será decidido pelos diretores do negócio. Existem vários modelos dessa ferramenta e várias estratégias usadas para definir o ROI. Os problemas da área de segurança da informação não podem ser medidos, o impacto causado por um problema é gigantesco e por isso essa segurança é necessária, para que anos de investimentos não vão por água abaixo devido a alguma falha na segurança.

\section{ Posicionamento Hierárquico}

Neste capítulo o autor trata de como deve haver uma mudança na hierarquia da empresa para que as novas demandas de segurança sejam atendidas. Ademais não se deve mais isolar as despesas para proteger de ameaças e impactos incididos aos ambientes somente na área de TI, o que não é correto, pois ele não é o único setor que deve ter atenção.

Por fim, o autor ainda afirma que deve existir um comitê de auditoria e segurança da informação junto com os executivos CIO, CEO e conselheiros, a fim de permitir ações estratégicas que estão de acordo com as diretrizes da empresa.

\section { Gerência de Mudanças}

Como o mundo está sempre mudando, é fácil que inovações tecnológicas e expansões afetem o ponto de equilibrio da empresa. Essas mudanças e dinamismos devem estar alinhados com os dinamismos da solução de segurança, pois esta não pode se manter estática perante as mudanças.

\section { Modelo de Gestão Corporativa de Segurança}

É um modelo que irá planejar toda a segurança da empresa. Ele é composto por várias etapas importantes que geram resultados significantes e que tendem a melhorar o resultado da próxima etapa. Assim, é possível facilmente reagir às mudanças que ocorrerão no negócio, diminuindo a chance do risco oscilar muito.

\section { Agregando Valor ao Negócio }

Todo esse processo traz um benefício gigante a empresa, pois além de evitar investimentos redudantes proporcionam uma melhor visão das falhas da empresa, além de esforços com o objetivo de melhorar a qualidade da empresa.

\bibliographystyle{sbc}
\bibliography{sbc-template}

\end{document}
